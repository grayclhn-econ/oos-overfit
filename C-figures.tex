\markright{}

\begin{figure}\centering
  {\large Coverage of DMW OOS interval for $\E_R\oosA$ in simulations}
  \tryinput{floats/mc-interval-testerror1.tex}
  \caption{Simulated coverage of $\E_R \oosA$ at 90\% confidence using
    a one-sided interval based on the DMW \oost\ test, plotted as a
    function of the fraction of observations used in the test sample,
    $P/T$. The solid horizontal line denotes the intervals' nominal
    coverage.}
 \label{fig:interval-R}
\end{figure}
\clearpage

\begin{figure}\centering
  {\large Coverage of DMW OOS interval for $\E_T\oosB$ in simulations}
  \tryinput{floats/mc-interval-generror1.tex}
  \caption{Simulated coverage of $\E_T \oosB$ at 90\% confidence using
    a one-sided interval based on the DMW \oost\ test, plotted as a
    function of the fraction of observations used in the test sample,
    $P/T$.  The solid horizontal line denotes the intervals' nominal
    coverage.}
  \label{fig:interval-T}
\end{figure}
\clearpage

\begin{figure}\centering
  {\large Size of DMW \oost\ test in simulations}
  \tryinput{floats/mc-dmwsize.tex}
  \caption{Simulated rejection probabilities for the DMW \oost\ test
    under the null hypothesis that the benchmark model will forecast
    better, $\E_T \oosB \leq 0$. The nominal size is 10\% and is
    marked with a solid horizontal line. Values greater than 10\%
    indicate that the test rejects the benchmark model too often. See
    Section~\ref{sec:simulation-design} for a discussion of the
    simulation design.}
  \label{fig:ttest-size}
\end{figure}
\clearpage

\begin{figure}\centering
  {\large Size of Clark-West OOS test in simulations}
  \tryinput{floats/mc-clarkwestsize.tex}
  \caption{Simulated rejection probabilities for Clark and West's
    (2006, 2007) OOS test statistic under the null hypothesis that the
    benchmark will forecast better, $\E_T \oosB \leq 0$. The nominal
    size is 10\% and is marked with a solid horizontal line. Values
    greater than 10\% indicate that the test rejects the benchmark
    model too often. See Section~\ref{sec:simulation-design} for a
    discussion of the simulation design.}
   \label{fig:clarkwest}
\end{figure}
\clearpage

\begin{figure}\centering
  {\large Size of McCracken \oost\ test in simulations}
  \tryinput{floats/mc-mccrackensize.tex}
  \caption{Simulated rejection probabilities for McCracken's (2007)
    \oost\ test under the null hypothesis that the benchmark model is
    more accurate, $\E_T \oosB \leq 0$. Nominal size is 10\% and is
    marked with a solid horizontal line. Values greater than 10\%
    indicate that the test rejects the benchmark model too often. See
    Section~\ref{sec:simulation-design} for a discussion of the
    simulation design.}
  \label{fig:mccracken}
\end{figure}
\clearpage

\begin{figure}\centering
  {\large Power of DMW \oost\ test in simulations}
  \tryinput{floats/mc-dmwpower.tex}
  \caption{Simulated rejection probabilities for the DMW \oost\ test
    under the alternative that the benchmark model is less accurate,
    $\E_T \oosB > 0$. Nominal size is 10\% and is marked with a solid
    horizontal line. Values greater than 10\% indicate that the test
    rejects the benchmark model too often. See
    Section~\ref{sec:simulation-design} for a discussion of the
    simulation design.}
  \label{fig:ttest-power}
\end{figure}
\clearpage

\begin{figure}
\centering
\large{Difference in OOS MSE of Prevailing Mean and\\ Kitchen
    Sink Models of Equity Premium (OLS)}
\tryinput{floats/empirics-oos-mse-1.tex}
\tryinput{floats/empirics-oos-mse-1b.tex}
\caption{OOS difference in the MSE
  of the prevailing mean benchmark and the kitchen sink model as a
  function of the test sample size, $R$.  Both models forecast the
  equity premium using annual data from 1928--2008.  The solid line
  gives the OOS average, and the shaded region indicates the
  one-sided 95\% confidence interval implied by the
  DMW test.  The bottom panel is a detailed view of the top
  panel for $R \geq 50$.}
\label{fig:empirics1}
\end{figure}
\clearpage

\begin{figure}
\centering
\large{Difference in OOS MSE of Prevailing Mean and\\ Kitchen
    Sink Models of Equity Premium (CT)}
\tryinput{floats/empirics-oos-mse-2.tex}
\tryinput{floats/empirics-oos-mse-2b.tex}
\caption{OOS difference in the MSE
  of the prevailing mean benchmark and the kitchen sink model as a
  function of the test sample size, $R$.  Both models forecast the
  equity premium using annual data from 1928--2008.  The solid line
  gives the OOS average, and the shaded region indicates the
  one-sided 95\% confidence interval implied by the
  DMW test.  The bottom panel is a detailed view of the top
  panel for $R \geq 50$.}
\label{fig:empirics2}
\end{figure}
\clearpage

\begin{figure}
\centering
\large{OOS MSE of Individual Forecasts of Equity Premium}
\tryinput{floats/empirics-oos-ind-ks.tex}
\tryinput{floats/empirics-oos-ind-pm.tex}
\caption{OOS MSE of the Prevailing Mean (PM) and
    Kitchen Sink (KS) models for equity premium prediction as
    a function of the size of the training sample, $R$.  Please note
    that the vertical scales are different in the two plots.}
\label{fig:empirics3}
\end{figure}

%%% Local Variables:
%%% mode: latex
%%% TeX-master: "paper"
%%% TeX-command-extra-options: "-shell-escape"
%%% End:
