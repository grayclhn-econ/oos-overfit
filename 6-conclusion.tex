\section{Conclusion}
\label{sec:conclusion}

This paper gives a theoretical motivation for using OOS comparisons:
the DMW OOS test allows a forecaster to conduct inference about
the expected future accuracy of his or her models when one or both is
overfit.  We show analytically and through Monte Carlo that standard
full-sample test statistics can not test hypotheses about this
performance.

Our paper also shows that popular test and training sample sizes may
give misleading results if researchers are concerned about overfit.
We show that $P^2/T$ must converge to zero for the DMW test to give
valid inference about the expected forecast accuracy, otherwise the
test measures the accuracy of the estimates constructed using only the
training sample.  In empirical research, $P$ is typically much larger
than this.  Our simulations indicate that using large values of $P$
with the DMW test gives undersized tests with low power, so this
practice may favor simple benchmark models too much.  Existing
corrections, proposed by \citet{ClM:01,ClM:05}, \citet{Mcc:07} and
\citet{ClW:06,ClW:07}, seem to correct too much, though, and reject
too often when the benchmark model is more accurate.

More work remains.  The requirement that $P^2/T$ converge to zero is
limiting, as it implies that in typical macroeconomic datasets, only a
handful of observations should be used for testing.  This requirement
can be relaxed only slightly; $P = O(T^{1/2})$ is required for the
OOS test to have nontrivial power in general, but there are loss
functions and DGPs for which some relaxation is possible.  This
constraint could be mitigated by extending our results to
cross-validation or other resampling strategies, or by constructing
full-sample statistics that allow inference about $\E_T \bar{D}_T$.
It would also be useful to extend our results to other forecasting
models and to explore how stationarity could be relaxed, but such
extensions are less important than improving the available statistics.

%%% Local Variables:
%%% mode: latex
%%% TeX-master: "paper"
%%% TeX-command-extra-options: "-shell-escape"
%%% End:
