\subsection{Proof of Theorem~\ref{res:insample1}}

Let $(\e_{t+h}, x_{t}) \sim i.i.d.\ N(0,I)$ and let $\theta_1 = 0$.
For any $d \geq 0$, the event $\E_T \oosB = -d$ implies that
\begin{equation*}
  \E_T (y_{T+\h+1} - x_{1T+1}'\bh{1T})^2
  = \E_T (y_{T+\h+1} - x_{2T+1}'\bh{2T})^2 - d \quad a.s.
\end{equation*}
which can be expressed as
\begin{equation}\label{eq:34}
  d + \theta_2'M_2' M_2 \theta_2 + \bh{1T}'\bh{1T} - \bh{2T}'M_1'M_1\bh{2T}
  = (\bh{2T} - \theta_2)' M_2'M_2 (\bh{2T} - \theta_2) \quad a.s.
\end{equation}
$M_2 \bh{2T}$ is normally distributed conditional on $\bh{1T}$ and
$M_1 \bh{2T}$, and is distributed on the surface of the sphere defined
by~\eqref{eq:34} conditional on $\bh{1T}$, $M_1 \bh{2T}$, and the
event $\E_T \oosB = -d$.

Since $M_2 \bh{2T}$ is normal, this conditional distribution is
invariant to reflection across the axes defined by the eigenvectors of
its covariance matrix. So when $\theta_2$ lies outside the cylinder
that contains the acceptance region of $\Lambda$, i.e. when
$\theta_2'M_2' V_T M_2 \theta_2 > c$, we have\footnote{%
  To keep the notation in these equations manageable, define the
  information set
  $\Gs = \sigma(\E_T \oosB \leq 0,~ \theta_2'M_2' V_T M_2 \theta_2 >
  c)$.} %
\begin{align*}
  \Pr[ \bh{2T}' M_2' V_T M_2 \bh{2T} \leq c \mid \Gs]
  &= \E\Big( \Pr[ \bh{2T}'M_2' V_T M_2 \bh{2T} \leq c \mid \Gs,~\bh{1T},~M_1 \bh{2T}] \mid \Gs \Big) \\
  & < 1/2
\end{align*}
since the inner conditional probability is less than 1/2.

Now let $\delta > 0$ be an arbitrary but small constant, and choose
$\theta_2$ far enough from the origin to ensure that
\begin{equation*}
  \Pr[\theta_2'M_2' V_T M_2 \theta_2 \leq c] \leq \delta
\end{equation*}
for large enough $T$. Then
\begin{align*}
  \E(\Lambda \mid \E_T \oosB \leq 0) &
  \geq \Pr[\bh{2T}' M_2' V_T M_2 \bh{2T} > c \mid \E_T \oosB \leq 0] \\
  &= \Pr[\bh{2T}' M_2' V_T M_2 \bh{2T} > c \text{~~and~~}
  \theta_2'M_2' V_T M_2 \theta_2 > c \mid \E_T \oosB  \leq 0] \\
  & \quad + \Pr[\bh{2T}' V_T M_2 \bh{2T} > c \text{~~and~~ }
  \theta_2'M_2' V_T M_2 \theta_2 \leq c \mid \E_T \oosB \leq 0].
\end{align*}
The second term is nonnegative by design. By conditioning on
$\theta_2'M_2' V_T M_2 \theta_2 > c$ our earlier argument shows that
the first term is greater than or equal to $(1 - \delta) / 2$. Since
$\delta$ is arbitrarily small, this completes the proof.
\qed

%%% Local Variables:
%%% mode: latex
%%% TeX-master: "paper"
%%% TeX-command-extra-options: "-shell-escape"
%%% End:
