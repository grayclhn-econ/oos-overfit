\section{Empirical analysis of equity premium predictability}
\label{sec:empirics}

This section presents a study of equity premium predictability based
on \citet{GoW:08}. Goyal and Welch conduct an out of sample analysis
of 18 different variables thought, based on previous research, to
predict the equity premium (calculated as the difference between the
return on the S\&P 500 index and the T-bill rate).\footnote{%
  \citet{GoW:08} builds on previous research by \citet{BoH:99} and
  \cite{GoW:03}.} %
Their analysis has two notable features: Goyal and Welch compare different
predictors across the same time periods and frequency as much as
possible, making the accuracy of these predictors directly comparable;
many of the papers that originally proposed these predictors used
different time periods, methods, or observational frequencies so their
results were not directly comparable. And Goyal and Welch compare
these predictors out-of-sample; many of the original studies found
in-sample evidence of equity premium predictability but did not look
at out-of-sample evidence.

Goyal and Welch consider many different models; most of them are very
simple---regression onto a constant and a single stochastic
predictor---but some are more complicated. They find that essentially
none of the models outperform a simple benchmark, the prevailing mean
of the equity premium, and conclude that these variables do not
predict the equity premium. A large literature has subsequently sought
to explain or rebut these results, including several responses to
Goyal and Welch's original working paper published in the same special
issue of \textit{The Review of Financial Studies} as
\citet{GoW:08}.\footnote{%
  Those papers are \citet{CaT:08}, \citet{Coc:08}, \citet{BRW:08}, and
  \citet{LeN:08}.} %

\citet{GoW:08}, as well as \citet{BoH:99}, \citet{LeN:08}, and many
other authors, propose that instability could explain the OOS failure
of these models. However, we have shown in this paper that overfit can
also explain this pattern, significant in-sample results that do not
hold up out of sample. In this section, we explore the extent to which
overfit is a potential concern in this data set and estimate the
expected forecasting performance of the largest model they consider, a
model with 13 regressors, using 81 observations (annual data from 1928
to 2009). The predictors are listed in Table~\ref{tab:equity}; please
see Goyal and Welch's original paper for detailed information about
these variables.\footnote{%
  Table~\ref{tab:equity} only lists the variables used in
  \citepos{GoW:08} ``kitchen sink'' model.  Some of the variables that
  they use are excluded from this model either because the series are
  too short or because the variables are linear combinations of other
  variables.} %

\citet{GoW:08} focus primarily on univariate regression models of the
form
\[
r_{t+1} = \beta_0 + \beta_1 x_{it} + \e_{t+1}
\]
using OLS with a recursive window, where $r_{t+1}$ is the equity
premium and $x_{it}$ is one of the predictors listed in
Table~\ref{tab:equity}.
This focus is consistent with much of the rest of the literature. But
it is not clear that this approach---using a large number of
restricted models---is any more reliable than using a single, large
model. Very few papers in the equity-premium predictability literature
explicitly account for the multiplicity of model comparisons\footnote{%
  \citet{RaW:06}, \citet{RaZ:12}, and \citet{Cal:15} are
  exceptions.} %
and most widely-used statistics for multiple comparisons are
derived under the assumption that there is a finite number of
hypotheses or models (as is the case in \citealp{STW:99c},
\citealp{Whi:00}, \citealp{Han:05}, and \citealp{LeR:05}), so there is
little theoretical evidence that they are any more reliable than a
regression model that encompasses all of the smaller models. Moreover,
since \citet{GoW:08} (along with most of the rest of the literature)
want to interpret these univariate regressions as meaningful
statements about the true relationships between the equity premium and
the regressors, omitted variable bias is a serious potential issue.

For example, Table~\ref{tab:gwinsample} presents two full-sample
coefficient estimates for each of the 12 predictors we
consider.\footnote{%
  For all of our full sample results, we studentize the
  regressors to make it easier to compare coefficient estimates and we
  express the equity premium in basis points.} %
The first column is the estimator of the coefficient on the
variable in the full regression,
\begin{equation*}
  r_{t+1} = \beta_0 + \sum_{i=1}^{12} \beta_{i} x_{it} + \e_{t+1},
\end{equation*}
and the third column is the $p$-value associated with a two-sided
$t$-test that the coefficient is zero in population. The second column
is the estimator of the coefficient from the univariate regression
\begin{equation*}
  r_{t+1} = \alpha_i + \gamma_{i} x_{it} + \e_{t+1},
\end{equation*}
and the fourth column is the $p$-value associated with its
$t$-test.\footnote{%
  All of the standard errors were calculated using a Newey-West kernel
  with two lags.} %
Some of the coefficient estimates agree, but some do not. The
coefficient estimate for the \emph{default yield spread}, for example,
is substantially and significantly negative in the full model ($-7.79$
with $p$-value $0.042$), but is near zero in the univariate regression
($0.79$ with $p$-value $0.720$). This pair of results implies that the
the default yield spread contains information about the equity premium
but is also correlated with other poor predictors, and that this
additional correlation adds noise to the univariate regression
and masks the true relationship. Similarly, \emph{net equity expansion}
is significant at 10\% in the univariate regression but not in the
full model ($p$-values of $0.060$ and $0.626$ respectively), which is
likely attributable to omitted variable bias. So there is merit to
studying a large model that includes all of the variables.

Results for the full sample estimates of these models are in
Table~\ref{tab:gwinsample}.\footnote{%
  Calculations in this section are done in R \citep{Rde:10} using the
  \emph{lmtest} \citep{ZeH:02} and \emph{sandwich} \citep{Zei:04}
  packages.}\footnote{%
  We compare the test statistic to critical values from the
  $F$-distribution, which have been shown to be more reliable than
  Chi-squared critical values when there are many regressors
  \citep{Ana:12,Cal:11c}.} %
The last rows of the table list measures of the full-model fit.  The
$p$-value for the test of full model fit is very small (less than
0.01), indicating that some of the coefficients are nonzero in
population and at least one of these predictors is correlated with the
equity premium.\footnote{%
  We have done additional analysis to try to identify which predictors
  were correlated with the equity premium, but none of the individual
  regressors were significant after correcting for multiplicity. The
  Bonferroni correction, for example, suggests that an individual
  $p$-value would need to be less than $0.10/12 \approx 0.0083$ for
  its corresponding coefficient to be significant at the 10\% level,
  but the smallest $p$-value is $0.035$. One can improve on the
  Bonferroni correction, of course, but a comprehensive analysis is
  beyond the scope of this paper.} %
As we argue throughout the paper, this result does not imply that the
model will forecast well.

To determine whether the full model can forecast well, we use the DMW
\oost\ test to compare it to a sample mean benchmark,
\begin{equation*}
r_{t+1} = \mu + \e_{1,t+1}.
\end{equation*}
Both models are estimated by OLS using the fixed-window scheme. We
also present results for a restricted model proposed by
\citet{CaT:08} that imposes that $\hat r_{t+1}$ be non-negative for
each forecast.
To study the effect of the training sample size on the DMW statistic, we
calculate the one-sided confidence interval for $\E_T \oosB$ given
by \eqref{interval-greater} corresponding to the null and alternative
hypotheses
\[ H_0: \quad E_T \oosB \leq 0 \qquad
H_A: \quad E_T \oosB > 0
\]
using the fixed-window scheme for each value of $R$ between 20 and
$T-10$. The standard deviation is estimated using a Newey-West
estimator with $\lfloor P^{1/4}\rfloor$ lags.  For small values of
$R$, the OOS average is expected to underestimate the performance of
the larger model relative to the smaller, but this may not hold in
this particular dataset. For the OOS results, we express the equity
premium in percentage points.

Figure~\ref{fig:empirics1} plots the OLS results and
Figure~\ref{fig:empirics2} imposes \citepos{CaT:08} restriction. The
solid line in each figure shows the OOS average, $\oosA$, and the
shaded region indicates the 95\% one-sided confidence interval implied
by the DMW test.  Negative numbers indicate that the full model has
higher out-of-sample loss.  We can see that the same patterns hold for
both models: the performance difference decreases as $R$ grows, but
the full model is never more accurate.  We also see that the
performance difference decreases suddenly over the period $R=29$ to
$R=34$ (corresponding to the years 1956--1961).
Figure~\ref{fig:empirics3} plots the accuracy of the individual
forecasts (only for the linear models) and shows that this change is
the result of a sudden improvement in the full model.  This change may
indicate instability in the underlying relationship, as proposed by
\citet{GoW:08}.

In summary, we fail to reject the null that the benchmark prevailing
mean model is more accurate than the full model including all of
\citepos{GoW:08} predictors.  This result is consistent with Goyal and
Welch's original analysis.  Unlike Goyal and Welch, we attribute this
result, at least in part, to parameter uncertainty---the full sample
results indicate that there is a true predictive relationship between
some of these variables and the equity premium and the larger model
could predict better than the benchmark with enough data.\footnote{%
  \citet{BWB:10} make a similar point about exchange rate models, but
  see also \citet{Chi:10} and \citet{Gia:10}.} %
These also indicate that combination or shrinkage estimators of the
full model have the potential to significantly improve on the
benchmark.\footnote{%
  See \citet{RaZ:12} for a recent review of this literature.} %

%%% Local Variables:
%%% mode: latex
%%% TeX-master: "paper"
%%% TeX-command-extra-options: "-shell-escape"
%%% End:
